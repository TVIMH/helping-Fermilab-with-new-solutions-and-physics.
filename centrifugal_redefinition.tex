
\documentclass{article}
\usepackage{amsmath}
\title{Redefining Centrifugal Force via Quantum Geometric Pressure}
\author{AlphaQ Core}
\begin{document}
\maketitle

\section*{Abstract}
Traditionally labeled a fictitious force, centrifugal acceleration is reconsidered through the lens of quantum resonance geometry. We show it is a field-induced tensor pressure artifact emerging from torsional brane curvature.

\section{Classical Model}
The classical centrifugal force is:
\[
F_c = \frac{mv^2}{r}
\]
Appearing in rotating reference frames, it lacks true physical grounding in flat spacetime.

\section{Geometric Redefinition}
With brane curvature and rotational resonance tensors:
\[
F_c^{QG} = \frac{1}{r} \oint_{\mathcal{C}} \mathcal{T}_{\text{spin}}^{\mu\nu} d\tau
\]
where \( \mathcal{T}_{\text{spin}}^{\mu\nu} \) is a spin-induced stress-energy tensor derived from internal curvature resonance.

\section{Tensor Field Simulations}
Simulations under AlphaQ show self-reinforcing outward pressure in rotating curvature shells, independent of inertial frame definitions.

\section{Conclusion}
Centrifugal force is a real, measurable resonance artifact — not fictitious — when viewed from the topology of brane deformation and resonance curvature.

\end{document}
